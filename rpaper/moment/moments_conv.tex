% Options for packages loaded elsewhere
\PassOptionsToPackage{unicode}{hyperref}
\PassOptionsToPackage{hyphens}{url}
\documentclass[
]{article}
\usepackage{xcolor}
\usepackage[margin=1in]{geometry}
\usepackage{amsmath,amssymb}
\setcounter{secnumdepth}{-\maxdimen} % remove section numbering
\usepackage{iftex}
\ifPDFTeX
  \usepackage[T1]{fontenc}
  \usepackage[utf8]{inputenc}
  \usepackage{textcomp} % provide euro and other symbols
\else % if luatex or xetex
  \usepackage{unicode-math} % this also loads fontspec
  \defaultfontfeatures{Scale=MatchLowercase}
  \defaultfontfeatures[\rmfamily]{Ligatures=TeX,Scale=1}
\fi
\usepackage{lmodern}
\ifPDFTeX\else
  % xetex/luatex font selection
\fi
% Use upquote if available, for straight quotes in verbatim environments
\IfFileExists{upquote.sty}{\usepackage{upquote}}{}
\IfFileExists{microtype.sty}{% use microtype if available
  \usepackage[]{microtype}
  \UseMicrotypeSet[protrusion]{basicmath} % disable protrusion for tt fonts
}{}
\makeatletter
\@ifundefined{KOMAClassName}{% if non-KOMA class
  \IfFileExists{parskip.sty}{%
    \usepackage{parskip}
  }{% else
    \setlength{\parindent}{0pt}
    \setlength{\parskip}{6pt plus 2pt minus 1pt}}
}{% if KOMA class
  \KOMAoptions{parskip=half}}
\makeatother
\usepackage{longtable,booktabs,array}
\usepackage{calc} % for calculating minipage widths
% Correct order of tables after \paragraph or \subparagraph
\usepackage{etoolbox}
\makeatletter
\patchcmd\longtable{\par}{\if@noskipsec\mbox{}\fi\par}{}{}
\makeatother
% Allow footnotes in longtable head/foot
\IfFileExists{footnotehyper.sty}{\usepackage{footnotehyper}}{\usepackage{footnote}}
\makesavenoteenv{longtable}
\usepackage{graphicx}
\makeatletter
\newsavebox\pandoc@box
\newcommand*\pandocbounded[1]{% scales image to fit in text height/width
  \sbox\pandoc@box{#1}%
  \Gscale@div\@tempa{\textheight}{\dimexpr\ht\pandoc@box+\dp\pandoc@box\relax}%
  \Gscale@div\@tempb{\linewidth}{\wd\pandoc@box}%
  \ifdim\@tempb\p@<\@tempa\p@\let\@tempa\@tempb\fi% select the smaller of both
  \ifdim\@tempa\p@<\p@\scalebox{\@tempa}{\usebox\pandoc@box}%
  \else\usebox{\pandoc@box}%
  \fi%
}
% Set default figure placement to htbp
\def\fps@figure{htbp}
\makeatother
\setlength{\emergencystretch}{3em} % prevent overfull lines
\providecommand{\tightlist}{%
  \setlength{\itemsep}{0pt}\setlength{\parskip}{0pt}}
\usepackage{bookmark}
\IfFileExists{xurl.sty}{\usepackage{xurl}}{} % add URL line breaks if available
\urlstyle{same}
\hypersetup{
  pdftitle={Moments Conversion Algorithm},
  pdfauthor={Abdullah Al Al Mahmud},
  hidelinks,
  pdfcreator={LaTeX via pandoc}}

\title{Moments Conversion Algorithm}
\author{Abdullah Al Al Mahmud}
\date{2023-11-23}

\begin{document}
\maketitle

\subsection{Abstract}\label{abstract}

Advantages

\begin{itemize}
\tightlist
\item
  Can be used for raw to raw and raw to and from central
\item
  always additions of positive quantities
\item
  Memorizing complex formulae is not required
\end{itemize}

Changing

\begin{itemize}
\tightlist
\item
  from one raw moment to another
\item
  raw to central
\item
  Central to Raw
\item
  R function to convert
\item
  Changing from function
\end{itemize}

\subsection{Introduction}\label{introduction}

We are going to employ the new method in four types of distribution.

\begin{itemize}
\tightlist
\item
  Raw data, for example (\(20, 25, 29, \cdots\))
\item
  Tabular distribution (value and corresponding probability)
\item
  PMF
\item
  PDF
\end{itemize}

\subsection{The Traditional Approach of Converting
Moments}\label{the-traditional-approach-of-converting-moments}

\subsubsection{Raw Moments to Other Raw
Moments}\label{raw-moments-to-other-raw-moments}

Let us revisit the traditiona method with the help of an example.

\textbf{Problem:} Let us suppose first three moments about 2 are -1, 7
\& 39. We have to find them about 5.

\textbf{Traditional Solution:}

Given

\begin{itemize}
\tightlist
\item
  \(\mu_1'(2) = -1\)
\item
  \(\mu_2'(2) = 7\) \&
\item
  \(\mu_3'(2) = 39\)
\end{itemize}

Now we have to find \(\mu_1'(5), \mu_3'(5), \text{ and } \mu_3'(5)\)

\paragraph{Conversion of First Moment}\label{conversion-of-first-moment}

We have: \[
\mu_1'(2) = \frac{1}{n} \sum_{i=1}^n (x_i - 2) = -1
\]

To find the first moment about 5, we rewrite the expression as: \[
\begin{aligned}
\mu_1'(5) &= \frac{1}{n} \sum_{i=1}^n (x_i - 5) \\
         &= \frac{1}{n} \sum_{i=1}^n \left[(x_i - 2) - 3\right] \\
         &= \frac{1}{n} \left( \sum_{i=1}^n (x_i - 2) - 3n \right) \\
         &= \mu_1'(2) - 3 = -1 - 3 = -4
\end{aligned}
\]

Thus, the first moment about 5 is \(\mu_1'(5) = -4\).

\subsubsection{Conversion of Second
Moment}\label{conversion-of-second-moment}

We are given: \[
\mu_2'(2) = \frac{1}{n} \sum_{i=1}^n (x_i - 2)^2 = 7
\]

To find the second moment about 5, we proceed as follows:
\begin{equation}
\begin{aligned}
\mu_2'(5) &= \frac{1}{n} \sum_{i=1}^n (x_i - 5)^2 \\
          &= \frac{1}{n} \sum_{i=1}^n \left[(x_i - 2) - 3\right]^2 \\
          &= \frac{1}{n} \sum_{i=1}^n \left[(x_i - 2)^2 - 6(x_i - 2) + 9\right] \\
          &= \frac{1}{n} \sum_{i=1}^n (x_i - 2)^2 - 6 \cdot \frac{1}{n} \sum_{i=1}^n (x_i - 2) + 9 \\
          &= \mu_2'(2) - 6\mu_1'(2) + 9 \\
          &= 7 - 6(-1) + 9 = 22
\end{aligned}
\end{equation}

Thus, the second moment about 5 is \(\mu_2'(5) = 22\).

\subsubsection{Conversion of Third
Moment}\label{conversion-of-third-moment}

We are given: \[
\mu_3'(2) = \frac{1}{n} \sum_{i=1}^n (x_i - 2)^3 = 39
\]

To compute the third moment about 5, we use the binomial expansion:
\begin{equation}
\begin{aligned}
\mu_3'(5) &= \frac{1}{n} \sum_{i=1}^n (x_i - 5)^3 \\
          &= \frac{1}{n} \sum_{i=1}^n \left[(x_i - 2) - 3\right]^3 \\
          &= \frac{1}{n} \sum_{i=1}^n \left[(x_i - 2)^3 - 9(x_i - 2)^2 + 27(x_i - 2) - 27\right] \\
          &= \frac{1}{n} \sum_{i=1}^n (x_i - 2)^3 
             - 9 \cdot \frac{1}{n} \sum_{i=1}^n (x_i - 2)^2 
             + 27 \cdot \frac{1}{n} \sum_{i=1}^n (x_i - 2) 
             - 27 \\
          &= \mu_3'(2) - 9\mu_2'(2) + 27\mu_1'(2) - 27 \\
          &= 39 - 9 \cdot 7 + 27 \cdot (-1) - 27 \\
          &= -78
\end{aligned}
\end{equation}

Therefore, the third moment about 5 is \(\mu_3'(5) = -78\).

Other higher order moments are converted in the same fashion, and the
calculation keeps getting more difficult.

\subsubsection{Raw to Central Moments}\label{raw-to-central-moments}

The following formulas are used:

\[
\begin{aligned}
\mu_1 &= 0 \\[6pt]
\mu_2 &= \mu_2'(a) - \left[\mu_1'(a)\right]^2 \\[6pt]
\mu_3 &= \mu_3'(a) - 3\mu_2'(a)\mu_1'(a) + 2\left[\mu_1'(a)\right]^3 \\[6pt]
\mu_4 &= \mu_4'(a) - 4\mu_3'(a)\mu_1'(a) + 6\mu_2'(a)\left[\mu_1'(a)\right]^2 - 3\left[\mu_1'(a)\right]^4 \\[6pt]
\mu_5 &= \mu_5'(a) - 5\mu_4'(a)\mu_1'(a) + 10\mu_3'(a)\left[\mu_1'(a)\right]^2
        - 10\mu_2'(a)\left[\mu_1'(a)\right]^3 + 4\left[\mu_1'(a)\right]^5
\end{aligned}
\]

\noindent Here, \(\mu_r'(a)\) denotes the \(r^{\text{th}}\) raw moment
about point \(a\) (an arbitrary number), and \(\mu_r\) denotes the
\(r^{\text{th}}\) central moment.

The traditional methods for transforming moments---either from one
arbitrary origin to another, or from raw to central moments---are often
tedious and procedurally distinct. While central moments can, in
principle, be derived by treating the arithmetic mean as a new origin,
this approach is computationally intensive, while computing central
moments using these formulas also poses a significant challenge, as it
requires careful handling of various combinations of raw moments of
differing orders. {[}REVIEW\ldots{]}

\subsubsection{Derivation of the novel
approach}\label{derivation-of-the-novel-approach}

Let,

Initial origin = \(a\)

Final origin = \(k\)

\paragraph{First moment}\label{first-moment}

First moment around \(a = \mu_1'(a)\)

First moment around \(k = \mu_1'(k)\)

\begin{equation} 
\begin{split}
\mu_1'(a) & =\frac{\sum(x_i-a)}{n} \\
\mu_1'(k) & = \frac{\sum(x_i-k)}{n}\\
          & = \frac{\sum(x_i-a-k+a)}{n}\\
          & = \frac{x_i-a}{n} + a - k\\
          & = \mu_1'(a) + (a - k)
\end{split}
\end{equation}

\paragraph{Second moment}\label{second-moment}

Second moment around \(a = \mu_2'(a)\)

Second moment around \(k = \mu_2'(k)\)

\begin{equation} 
\begin{split}
\mu_2'(a) & =\frac{\sum(x_i-a)^2}{n} \\
\mu_2'(k) & = \frac{\sum(x_i-k)^n}{n}\\
          & = \frac{\{\sum(x_i-a-k+a)^2+2(a-k)(x_I-a)+(a-k)^2\}}{n}\\
          & = \frac{(x_i-a)^2}{n} + 2(a - k)\frac{\sum(x_i-a)}{n}+(a-k)^2\\
          & = \mu_2'(a) + 2(a - k)\mu_1'(a)+(a-k)^2
\end{split}
\end{equation}

Other moments are dealt with in the same fashion.

To generalize the process, let

\begin{itemize}
\tightlist
\item
  \(b = a - k\) (for a reason to be obvious soon),
\item
  \(\mu_r'(a) =\) \(r^{\text{th}}\) raw moment about \(a\) and
\item
  \(\mu_r'(k) =\) \(r^{\text{th}}\) raw moment about \(k\)
\end{itemize}

So we have:

\begin{itemize}
\tightlist
\item
  \(\mu_1'(k) = \mu_1'(a)+b\)
\item
  \(\mu_2(k)= \mu_2'(a) + 2b \space \mu_1'(a)+b^2\)
\item
  \(\mu_3'(k) = \mu_3'(a) + 3b \space \mu_2'(a) + 3b^2 \space \mu_1'(a)  + b^3\)
\item
  \(\mu_4'(k) = \mu_4'(a) + 4b \space \mu_3'(a) + 6b^2 \space \mu_2'(a)  +4b^3 \space \mu_1'(a) + b^4\)
\end{itemize}

The formulae look a lot like binomial expansions (which is natural due
to the formula of the moments themselves). One modification will make
them look (also serving our purpose) perfectly like binomial expansions.

We let \(\mu_r'(a) = a^r\)

Which simply means whatever the power of \(a\), we are dealing with that
particular moment.

Thus letting the terms of binomial expansion \(a\) and \(b\), then for
converting the first moment, we use the formula of \((a+b)^1 = a + b\).
The table below shows the formulae required to convert the first four
moments:

\begin{longtable}[]{@{}cc@{}}
\toprule\noalign{}
Moment & Formula \\
\midrule\noalign{}
\endhead
\bottomrule\noalign{}
\endlastfoot
First & \(a+b\) \\
Second & \(a^2+2ab+b^2\) \\
Third & \(a^3+3a^2b+3ab^2+b^3\) \\
Forth & \(a^4+4a^3b+6a^2b^2+4ab^3+b^4\) \\
\end{longtable}

Observe the coefficients. They are essentially the binomial
coefficients. The coefficients can be obtained from the Pascal's
triangle, which is a triangular array of the binomial coefficients.

\[
{\begin{array}{c}
1 \\
1 \quad 1 \\
1 \quad 2 \quad 1 \\
1 \quad 3 \quad 3 \quad 1 \\
1 \quad 4 \quad 6 \quad 4 \quad 1 \\
1 \quad 5 \quad 10 \quad 10 \quad 5 \quad 1 \\
1 \quad 6 \quad 15 \quad 20 \quad 15 \quad 6 \quad 1 \\
1 \quad 7 \quad 21 \quad 35 \quad 35 \quad 21 \quad 7 \quad 1 \\
\end{array}}
\]

The first row is 1. Each entry of each subsequent row is constructed by
adding the number above and to the left with the number above and to the
right, treating blank entries as 0. For example, the initial number of
row 1 (or any other row) is 1 (the sum of 0 and 1), whereas the numbers
1 and 3 in row 3 are added to produce the number 4 in row 4.

\subsubsection{Testing a conversion}\label{testing-a-conversion}

From the previous problem,

\(b = a-k = 2-5=-3\)

\subsubsection{First Moment}\label{first-moment-1}

We need to find \(\mu_1'(5)\)

\[
\begin{align*}
\mu_1'(5) &= a + b \\
&= \mu_1'(2) + b \\
&= -1 + (-3) \\
&= -4
\end{align*}
\]

\subsubsection{Second Moment}\label{second-moment-1}

\[
\begin{align*}
\mu_2'(5) &= a^2 + 2ab + b^2 \\
&= \mu_2'(2) + 2\mu_1'(2)b + b^2 \\
&= 7 + 2(-1)(-3) + 9 \\
&= 7 + 6 + 9 \\
&= 22
\end{align*}
\]

\subsubsection{Third Moment}\label{third-moment}

\(\mu_3'(5) = 39 + 3(-3)\times 7 + 3 (-3)^2(-1) + (-3)^3 = -78\)

\subsubsection{Centrtal Moments from Raw
Moments}\label{centrtal-moments-from-raw-moments}

We know, the central moments are nothing but moments around arithmetic
mean (\(\bar X\))

The rth central moment is \(\mu_r = \frac{\sum (x_i-\bar x)^r}{n}\)

The central moments (\(\mu_r\)) can be derived from the raw moments. The
conventional, formulae, however, look esoteric and difficult to
remember.

\begin{itemize}
\tightlist
\item
  \(\mu_1 = 0\)
\item
  \(\mu_2 = \mu_2'- a\)
\item
  \(\mu3 = \mu_3'- 3 \mu_2' \mu_1' + 2 \mu_1'^3\)
\item
  \(\mu4 = \mu_4' - 4 \mu_3' \mu_1' + 6 \mu_2' \mu_1'^2 - 3 \mu_1'^4\)
\end{itemize}

But we can derive them quickly and easily by means of binomial theorem,
and we can get the coefficient from Pascal's triangle.

\subsubsection{Moments of Binomial
Distribution}\label{moments-of-binomial-distribution}

Let \(X \sim \text{Bin}(n, p)\). The moment generating function (MGF) is
defined as:

\[
M_X(t) = \mathbb{E}[e^{tX}] = \sum_{x=0}^{n} e^{tx} \binom{n}{x} p^x (1-p)^{n-x}
\]

Factor out \(e^{tx}\):

\[
M_X(t) = \sum_{x=0}^{n} \binom{n}{x} (p e^t)^x (1-p)^{n-x}
\]

This is a binomial expansion of:

\[
M_X(t) = \left( (1 - p) + p e^t \right)^n
\]

\paragraph{First Moment from MGF (about origin or
Zero)}\label{first-moment-from-mgf-about-origin-or-zero}

Differentiate the MGF with respect to \(t\):

\[
M'_X(t) = \frac{d}{dt} \left[ (1 - p + p e^t)^n \right]
= n (1 - p + p e^t)^{n-1} \cdot p e^t
\]

Evaluate at \(t = 0\) to get the first moment around zero (\(\mu_1\) or
mean):

\[
M'_X(0) = n (1 - p + p)^{n - 1} \cdot p = n p
\]

\textbf{Therefore, the first moment (mean) is:}

\[
\boxed{\mathbb{E}[X] = n p}
\]

\paragraph{First Central Moment of Binomial
Distribution}\label{first-central-moment-of-binomial-distribution}

\[
M_{X - \mu}(t) = e^{- \mu t} \cdot M_X(t)
\]

Substituting \(\mu = np\):

\[
M_{X - \mu}(t) = e^{-np t} \cdot \left( (1 - p) + p e^t \right)^n
\]

To find the \textbf{first central moment}, we differentiate this and
evaluate at \(t = 0\):

\[
\mu_1 = \left. \frac{d}{dt} M_{X - \mu}(t) \right|_{t = 0}
\]

Let:

\[
A(t) = e^{-np t}, \quad B(t) = \left( (1 - p) + p e^t \right)^n
\]

Then:

\[
M_{X - \mu}(t) = A(t) \cdot B(t)
\]

Differentiate using the product rule:

\[
\frac{d}{dt}(A(t)B(t)) = A'(t)B(t) + A(t)B'(t)
\]

Now evaluate each part at \(t = 0\):

\begin{itemize}
\tightlist
\item
  \(A(0) = 1\),\\
\item
  \(A'(t) = -np e^{-np t} \Rightarrow A'(0) = -np\),\\
\item
  \(B(0) = \left( (1 - p) + p \cdot 1 \right)^n = 1\),\\
\item
  \(B'(t) = n \left( (1 - p) + p e^t \right)^{n - 1} \cdot p e^t \Rightarrow B'(0) = n p\)
\end{itemize}

So:

\[
\left. \frac{d}{dt}(A(t)B(t)) \right|_{t = 0}
= A'(0) B(0) + A(0) B'(0) = (-np)(1) + (1)(np) = 0
\]

\begin{center}\rule{0.5\linewidth}{0.5pt}\end{center}

The \textbf{first central moment} of the Binomial distribution is
\textbf{zero}, as expected, although we had to go through a tedious
process. Using the novel method, however, we can find the same quickly
and more easily.

The first central moment (\(\mu_1\)) can also be written as
\(\mu_1'(\mu)\) for clarity for the purpose of using the new technique.

Thus we have

\[
\begin{align*}
\text{Initial origin:} \quad & a =  \\
\text{New origin:} \quad & k = \mu = np \\
\text{Shift:} \quad & b = a - k =  - np = -np \\
\text{First central moment:} \quad & \mu_1 = \mu_1'(a) + b = \mu_1'() + (-np) = np - np = 
\end{align*}
\]

Which gave us the expected value (0, literally as well) with ease.

Similarly we can find the second central moment.

We know the second raw moment of Binomial distribution about zero is
\(\mu_2'(0) = np - np^2 + n^2p^2\)

\[
\begin{align*}
\mu_2 &= \mu_2'(np) = \mu_2'(\mu) \\
&= a^2 + 2ab + b^2 \\
&= \mu_2'(0) + 2\mu_1'(0)b + b^2 \\
&= np + n^2p^2 - np^2 + 2(np)(-np) + (-np)^2 \\
&= np + n^2p^2 - np^2 - 2n^2p^2 + n^2p^2 \\
&= np - np^2 \\
&= np(1-p)
\end{align*}
\]

We again obtain the correct moment.

\subsubsection{Normal Distribution}\label{normal-distribution}

First four central and raw moments about zero are

\begin{longtable}[]{@{}
  >{\centering\arraybackslash}p{(\linewidth - 6\tabcolsep) * \real{0.1548}}
  >{\centering\arraybackslash}p{(\linewidth - 6\tabcolsep) * \real{0.4762}}
  >{\centering\arraybackslash}p{(\linewidth - 6\tabcolsep) * \real{0.2024}}
  >{\centering\arraybackslash}p{(\linewidth - 6\tabcolsep) * \real{0.1667}}@{}}
\toprule\noalign{}
\begin{minipage}[b]{\linewidth}\centering
Raw Moments
\end{minipage} & \begin{minipage}[b]{\linewidth}\centering
\end{minipage} & \begin{minipage}[b]{\linewidth}\centering
Central Moments
\end{minipage} & \begin{minipage}[b]{\linewidth}\centering
\end{minipage} \\
\midrule\noalign{}
\endhead
\bottomrule\noalign{}
\endlastfoot
Notation & Value & Notation & Value \\
\(\mu_1'\) & \(\mu\) & \(\mu_1\) & \(0\) \\
\(\mu_2'\) & \(\mu^2 + \sigma^2\) & \(\mu_2\) & \(\sigma^2\) \\
\(\mu_3'\) & \(\mu^3 + 3\mu \sigma^2\) & \(\mu_3\) & \(0\) \\
\(\mu_4'\) & \(\mu^4 + 6\mu^2 \sigma^2 + 3 \sigma^4\) & \(\mu_4\) &
\(3 \sigma^4\) \\
\end{longtable}

\paragraph{Using the Novel Approach}\label{using-the-novel-approach}

\(b = a- k = 0 - \mu = -\mu\)

\(\mu_1 = a + b = \mu_1' + b = \mu - \mu = 0\)

\[
\begin{align*}
\mu_2 &= a^2 + 2ab + b^2 \\
&= \mu_2' + 2\mu_1'b + b^2 \\
&= (\mu^2 + \sigma^2) + 2\mu(-\mu) + \mu^2 \\
&= \mu^2 + \sigma^2 - 2\mu^2 + \mu^2 \\
&= \sigma^2
\end{align*}
\]

Similarly the third central moment

\[
\begin{align*}
\mu_3 &= a^3 + 3a^2b + 3ab^2 + b^3 \\
&= \mu_3' + 3\mu_2'b  3\mu_1'b^2 + b^3 \\
&= (\mu^3 + 3\mu \sigma^2) + 3(\mu^2+\sigma^2) (-\mu) + 3\mu(-\mu)^2 +  + (-\mu)^3 \\
&= 0
\end{align*}
\]

We get 0 as expected, since all odd central moments of the normal
distribution is zero.

\end{document}
