\documentclass{article}
\usepackage{hyperref}
\usepackage{amsmath}
\usepackage{amssymb}
\usepackage{enumitem}
\usepackage{etoolbox}

% Define a command for hints
\newcommand{\hint}[1]{%
    \par\noindent\textbf{Hint:} #1%
    \par\vspace{\baselineskip}%
}

% Define a global token list to store solutions
\newtoks\globalsolutions
\globalsolutions{} % Initialize the token list

% Command to add solutions to the global list
\newcommand{\addtoglobalsolutions}[1]{%
    \global\globalsolutions=\expandafter{\the\globalsolutions\item #1}%
}

% Remove trailing periods from section and subsection numbers
\renewcommand{\thesection}{\arabic{section}}
\renewcommand{\thesubsection}{\arabic{subsection}}

% Custom label format for questions
\renewcommand{\labelenumi}{\textbf{Question \thesection.\thesubsection.\arabic{enumi}}}

% Command to display all solutions at the end
\newcommand{\printsolutions}{%
    \section*{Solutions}%
    \begin{enumerate}[label=\textbf{Solution \thesection.\thesubsection.\arabic*}]%
        \the\globalsolutions% Expand the token list
    \end{enumerate}%
}

\title{Questions and Solutions}
\author{Your Name}
\date{\today}

\begin{document}

\maketitle

\tableofcontents

\newpage

\section{Mathematical Problems}
\subsection{Velocity}

\begin{enumerate}
    \item A train leaves New York at 8:00 AM traveling west at 60 mph. Another train leaves Los Angeles at 6:00 AM traveling east at 70 mph on the same track. If the distance between New York and Los Angeles is 2,800 miles, at what time will the two trains meet?
    \hint{Use the relative speed concept to determine the time when the two trains meet.}
    \addtoglobalsolutions{The two trains will meet at 4:28 AM the next day.}
    
    \item Solve for \(x\) in the equation \(x^2 - 4 = 0\).
    \hint{Factor the equation or use the quadratic formula.}
    \addtoglobalsolutions{The solutions are \(x = 2\) and \(x = -2\).}
\end{enumerate}

\section{Topic}
\subsection{Subtopic}

\begin{enumerate}
    \item What is the derivative of \(f(x) = x^3\)?
    \hint{Use the power rule for differentiation.}
    \addtoglobalsolutions{The derivative is \(f'(x) = 3x^2\).}
    
    \item Evaluate the integral \(\int x \, dx\).
    \hint{Use the power rule for integration.}
    \addtoglobalsolutions{The integral evaluates to \(\frac{x^2}{2} + C\).}
\end{enumerate}

\newpage

% Print all solutions at the end
\printsolutions

\end{document}